\subsection{Consequências do Overfitting e Underfitting}

As implicações práticas do \textit{overfitting} e do \textit{underfitting} vão além de métricas acadêmicas: elas impactam diretamente a tomada de decisão, a confiança dos usuários e os custos associados ao desenvolvimento e manutenção de sistemas de IA. Esta subseção explora as principais consequências desses problemas, destacando por que compreender e mitigá-los é essencial em aplicações reais.

\subsubsection{Impacto na tomada de decisões erradas em ambientes críticos}
Quando modelos sofrem de overfitting ou underfitting em ambientes críticos — como diagnóstico médico, sistemas de crédito, detecção de intrusão ou segurança pública — as consequências são muito reais: falsos positivos ou negativos, diagnósticos incorretos, liberação de crédito quando não se deveria ou bloqueio indevido de usuários, falhas de segurança. Por exemplo, um modelo de detecção de fraude que não generaliza adequadamente pode permitir fraudes novas, gerando perdas financeiras e vazamentos de dados \cite{zhalgas2025robustfacerecognition}.

\subsubsection{Perda de confiança em sistemas de IA}
Se o modelo começa a falhar repetidamente (por exemplo, recomendando produtos irrelevantes ou fazendo mau reconhecimento facial), os usuários perdem confiança no sistema. Uma recomendação que sempre “acerta” no histórico mas falha para novos usuários gera frustração, menor engajamento e possível abandono da plataforma.

\subsubsection{Tempo e recursos gastos com modelos mal ajustados}
Modelos que estão mal ajustados — seja por serem \textit{overfit} ou \textit{underfit} — geram retrabalho: mais tempo investido em ajustar hiperparâmetros, recolher dados, re-treinar modelos, realizar validações extras, corrigir falhas em produção. Isso implica custos adicionais de hardware, equipe, manutenção e possíveis prejuízos por decisões incorretas.
