\subsection{Troca de Viés-Variância}

O \textbf{viés} diz respeito à diferença existente entre as previsões médias do modelo e os valores corretos previstos. Assim, modelos com alto viés acabam introduzindo erros por conta de suposições simplificadas e tendem ao \textit{underfitting}.

A \textbf{variância}, por outro lado, representa o quanto as previsões de um modelo mudam para diferentes conjuntos de dados de teste. Modelos com alta variância são mais sensíveis a ruídos e tendem ao \textit{overfitting}.

Essas duas características são inversamente proporcionais e o desafio central para um bom modelo de \textit{machine learning} é encontrar um ponto de equilíbrio entre ambas, conhecido como \textbf{troca de viés-variância} (ou \textit{bias-variance trade-off}), em que se busca criar um modelo que seja suficientemente complexo para capturar padrões, mas simples o bastante para generalizar.