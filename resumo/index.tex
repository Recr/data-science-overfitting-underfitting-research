\keywords{Overfitting; Underfitting; Machine Learning; Generalização; Regularização.}
\renewcommand{\abstractname}{Resumo}
\begin{abstract}
	Este trabalho investiga os fenômenos de \textit{overfitting} e \textit{underfitting} em modelos de aprendizado de máquina, destacando suas causas, implicações práticas e estratégias de mitigação. São discutidos conceitos fundamentais como generalização, troca viés–variância, validação cruzada, curvas de aprendizado e técnicas de regularização. Exemplos reais em domínios como reconhecimento facial, detecção de fraude e previsão de demanda ilustram como modelos mal ajustados impactam diretamente a confiabilidade e o custo operacional de sistemas de IA. A partir disso, o estudo apresenta abordagens consolidadas para mitigar esses problemas, além de analisar o estado atual do tema no meio acadêmico e na indústria, ressaltando a crescente relevância de modelos sobreparametrizados e os desafios teóricos em torno da generalização. O trabalho conclui apontando a necessidade de aprofundar a compreensão sobre esses fenômenos para garantir modelos robustos e responsáveis em aplicações críticas.
\end{abstract}


\keywords{Overfitting; Underfitting; Machine Learning; Generalization; Regularization.}
\renewcommand{\abstractname}{Abstract}
\begin{abstract}
	This work examines the phenomena of overfitting and underfitting in machine learning models, emphasizing their causes, practical implications, and mitigation strategies. Fundamental concepts such as generalization, the bias–variance trade-off, cross-validation, learning curves, and regularization techniques are discussed. Real-world examples from facial recognition, fraud detection, and demand forecasting demonstrate how poorly adjusted models directly affect the reliability and operational costs of AI systems. The study then presents consolidated approaches to mitigate these issues and analyzes the current state of the field in both academia and industry, highlighting the growing relevance of overparameterized models and ongoing theoretical challenges related to generalization. The paper concludes by underscoring the need for deeper research to ensure robust and responsible models in critical applications.
\end{abstract}
