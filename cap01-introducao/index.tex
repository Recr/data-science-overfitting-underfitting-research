\section{Introdução}

Com os avanços da tecnologia digital, interações online têm sido cada vez mais comuns no cotidiano. Redes sociais, plataformas de \textit{streaming}, lojas virtuais e vários outros ambientes digitais são responsáveis por gerar volumes massivos de dados que crescem conforme o passar dos anos. Estima-se, inclusive, que ao fim de 2025 o volume global de dados aumente para 181 zettabytes de acordo com o Statista \cite{rivery2025}.

Esse cenário demonstra-se muito promissor para a área de Inteligência Artificial (IA), especialmente para o ramo de \textit{machine learning} (ML) no qual o objetivo é analisar e encontrar algum tipo de valor nessa vasta quantidade de dados. No entanto, para que isso seja possível, é necessário encontrar um equilíbrio entre generalização e aprendizado.

Diante disso, surgem os conceitos de \textit{overfitting} (“sobreajuste”) e \textit{underfitting} (“subajuste”), que correspondem respectivamente a um modelo de ML que aprende demais ou de menos de forma a impactar negativamente em sua performance. 

O presente estudo tem por objetivo, portanto, explorar esses conceitos relacionados ao contexto de \textit{machine learning}, focando na problemática do tema, formas de mitigação e — por fim — uma análise sobre o estado atual do tema.